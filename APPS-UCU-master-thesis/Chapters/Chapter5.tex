\chapter{Conclusion}

In this work, we were researching possibilities of effective application of probabilistic approaches described at DeepAR [\cite{2017arXiv170404110S}] in a field of AutoML using DQN. As a result of the research, we've come with a solution that requires the implementation of a Gaussian Layer to master CNN and usage of MLE of variance as a loss function. A proposed solution was validated on a set of experiments showing results compared to the original algorithm and outperformed it during transfer learning validation.

The solution is available on the GitHub [\cite{github}]. Note that this code is heavily inspired and build on top of this two projects [\cite{nasgithub}; \cite{rltutorial}].

We hope that this example will show that AutoML research using RL is already a thing and could be held by other students and researches even on an occasion of low resources available. We hope to see more and more research in the filed in the upcoming years.

A few days before I was writing this conclusion NAS-Bench-102 was published. It introduced different search space, results on multiple datasets, and more diagnostic information [\cite{nas102}]. As the community is building up this ecosystem we have no doubt NAS would become a problem solved during challenges, and, what's more important, more reproducible. This, however, opens new challenges for us, and one of the biggest ones would be to move existing codebase of the project to NASbench.
